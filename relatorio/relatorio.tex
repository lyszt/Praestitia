\documentclass[a4paper]{article}

\usepackage[utf8]{inputenc}
\usepackage[T1]{fontenc}
\usepackage{inter}
\usepackage{parskip}
\usepackage{graphicx} 
\usepackage{setspace}
\usepackage{lettrine}
\usepackage{fancyhdr}
\usepackage[portuguese]{babel}
\usepackage[dvipsnames]{xcolor}
\usepackage{grffile}
\usepackage{minted}

\renewcommand{\baselinestretch}{1.5} 

\fancyhf{} 
\renewcommand{\headrulewidth}{0pt} 
\rfoot{\thepage}
\definecolor{LightGray}{gray}{0.9}
\pagestyle{fancy}

\color{darkgray}
\usemintedstyle{pastie}

\begin{document}
\renewcommand{\contentsname}{Sumário}
\tableofcontents
\newpage

\section{Introdução}

\subsection{Comanda do trabalho}
A Abensoft é uma empresa de tecnologia que se especializa no desenvolvimento de software para o mercado imobiliário. A proposta central da companhia é fornecer soluções que integrem os diferentes profissionais que atuam no setor. O software da Abensoft é direcionado a um público profissional que inclui correspondentes bancários, corretores de imóveis, imobiliárias e construtoras.

Um correspondente bancário, no contexto imobiliário, é uma empresa que atua como intermediária para instituições financeiras, facilitando e agilizando processos de financiamento.

\end{document}
